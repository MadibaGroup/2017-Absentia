% !TEX root = ../main.tex
\section{Introductory Remarks}

Threat model: secret inputs, correct computation


No secrets, trust computation
- cloud computing problem

No secrets, no trust
- verifiable computing

one secret, no trust
- verifiable computing

two or more secrets, trust
- honest but curious model - two GC, three+ MPC (based on secret sharing)

two or more secrets, no trust
- malicious model - same as above, plus mix and match 

public verifiable
- mix and match 

One secret, no trust, public verifiable 
FHE (more than one secret - who has decryption key?) 
Joint decryption at the end for MPC

Functional encryption
- seems different

\subsection{Deployability}

Multi users with secrets

Users to not have to be online at the same time (excludes GC, MPC, etc) - one round of communication to submit inputs

they will have to submit somehow: minimize what they need to know at submission time. encrypt it but need to know a key - to know key, need to know computational trustees

minimize privacy assumptions around submission - submit to a public broadcast channel (as opposed to private channels with trustees) 

at the end, they have to receive the output - public channel, asynchronous, one round of communication, 

at the end, they have to check a proof - great to delegate proof to others so they don't have to check (public verifiability vs normal malicious model)

\subsection{Mix and Match}

There is only one SFE protocol that seems to work this way out of the box: Mix and Match.

Users, Trustees, Miners

\subsection{Trustees}

Why did mix and match fail? Who are the trustees going to be? IBM, Microsoft, Google in 2-out-of-3 is great for papers but not realistic 

Anyone can be with some human verification on top - Tor model 

Plus, unlike Tor, we can pay them

\subsection{Coordination}

Bulletin board: append-only (immutable) broadcast (non-equivocating) - blockchain
Payments: blockchain
Verification for free: blockchain

Which blockchain? Ethereum
Other related projects: Enigma

 

