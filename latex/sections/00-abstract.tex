% !TEX root = ../main.tex
\begin{abstract}

This paper describes a blockchain-based approach for secure function evaluation in the setting where multiple participants have private inputs that no other individual should learn. The emphasis of Absentia is reducing the participants' work to a bare minimum, where they can effectively have the computation performed in their absence and they can trust the result. Our use of Ethereum is not necessary --- the underlying SFE protocol, based on Mix and Match, can operate perfectly well without a blockchain. However Ethereum does add value in the three following ways: (1) SFE requires a secure bulletin board and blockchains are the most widely deployed data structure with bulletin board properties (immutability and non-equivocation). (2) Ethereum provides a built-in mechanism to financially compensate participants for the work they perform. (3) A publicly verifiable SFE protocol can be checked by the Ethereum network itself, absolving the user of having to verify that the function was executed correctly.

\end{abstract}