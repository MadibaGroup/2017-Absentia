\section{Related work}
* Enigma~\cite{zyskind2015enigma}

uses secure multiparty computation

utilize the blockchain to manage (i) access control, (ii) identities and (iii) to log events in a tamper-proof way.

Transactions in the blockchain are not financial, they contain instructions for storing, querying and sharing data.

proof of correct computation is recorded on chain
\newline
\newline
* Decentralizing Privacy: Using Blockchain to Protect Personal Data ~\cite{zyskind2015decentralizing}
\newline
\newline
* Zexe~\cite{bowe2020zexe}

uses succinct zero knowledge proofs and recursive proof composition

not Ethereum- based
\newline
\newline
* MPC on Bitcoin~\cite{andrychowicz2014secure}

ensure honest behavior among the MPC participants by penalizing the dishonest par- ticipants.

A person who deviates from the protocol has to pay a fee if it does not reveal the secret value in a determined time interval. 
\newline
\newline
* Mix-and-match~\cite{jakobsson2000mix}
Our system is based on mix-and-match protocol

a new approach to secure multiparty computation that does not heavily rely on verifiable secret sharing. 

It is based on creating a boolean representation of the computation to be done and representing it in a table. 

Private key is shared among the participants for working on the ciphertexts. 

The participants should decide on a function and its circuit representation before they proceed with the mix and match protocol.

Mix and match protocol guarantees that the participants only learn the output of their jointly chosen function and nothing else.

* How to Use Bitcoin to Incentivize Correct Computations,Ranjit Kumaresan and Iddo Bentov

* Hawk~\cite{kosba2016hawk}

transaction privacy

privacy-preserving smart contracts

Hawk is responsible for compiling the program written by the user into a cryptographic protocol between them and the blockchain.

A use case: fair MPC

* HoneybadgerMPC~\cite{lu2019honeybadgermpc}


* SPDZ ~\cite{damgaard2013practical}

~\textbf{Mix-and-match vs SPDZ}
\section{System design}

~\textbf{Alice:} can call loadFunds more than once. loadCiphertexts  x 4

~\textbf{Bob:} can call loadFunds more than once. loadCiphertexts x 4

~\textbf{Trustee1:} is also the manager. Trustee1 calls  
\begin{enumerate}
	\item loadOutputs  x 2
	\item  createRow1  x 4
\end{enumerate}
at the beginning of the protocol. 
\newline

The following is repeated for each PET (8 times): 
\begin{enumerate}
	\item PET\_subtract
	\item DHProve\_Trustee1Pf\_Rand
	\item DHProve\_Trustee1Pf\_PartialDec
	 \item FullDecryption
\end{enumerate}

After all PETs are run, Trustee1 has to run the following:
\begin{enumerate}
	\item matchingRow
	\item matchingValue
	\item createFinalDecryption
	\item loadFinalCiphertext
	\item DHProve\_Trustee1Pf\_PartialDec
	\item FullDecryption
\end{enumerate}

~\textbf{Trustee2:} 

The following is repeated for each PET (8 times): 
\begin{enumerate}
	\item DHProve\_Trustee2Pf\_Rand 
	\item DHProve\_Trustee2Pf\_PartialDec
\end{enumerate}
and calls DHProve\_Trustee2Pf\_PartialDec once for the final decryption.


\begin{table}[]
	\begin{tabular}{|l|l|l|l|}
		\hline
		Contract & Function  & Gas  & Gas Cost (\$)  \\ \hline
		\multirow{9}{*}{}  
		& loadFund & 28040  &1.50  \\ \cline{2-4} 
		& checkBalances &  30359& 1.63 \\  \cline{2-4} 
		&  loadOutputs&  300798 & 16.13 \\  \cline{2-4} 
		&  createRow1 & 8741453 & 468.90 \\  \cline{2-4}
		Mixmatch.sol	&  matchingRow & 37547 &2.01  \\  \cline{2-4}
		& matchingValue &  40868&2.19 \\  \cline{2-4}
		&  createFinalDecryption&  4430611& 237.66 \\  \cline{2-4}
		& sendexcessfunds &  41110& 2.21 \\  \cline{2-4}
		& withdraw &  39221& 2.10 \\   \hline
		\multirow{8}{*}{} 
		&  loadCiphertexts & 304668 & 16.34 \\  \cline{2-4}
		& PET\_subtract & 242131 & 12.99\\  \cline{2-4}
		&  DHProve\_Trustee1Pf\_Rand& 815340 &  43.74\\  \cline{2-4}
		PET.sol	&  DHProve\_Trustee2Pf\_Rand& 393561 &21.11  \\  \cline{2-4}
		&  DHProve\_Trustee1Pf\_PartialDec& 364298 & 19.54\\  \cline{2-4}
		& DHProve\_Trustee2Pf\_PartialDec & 363612 & 19.50 \\  \cline{2-4}
		& FullDecryption  &  107086& 5.74 \\  \cline{2-4}
		&  loadFinalCiphertext& 173945 & 9.33\\ \hline
	\end{tabular}
	\caption{Table description!! (Gas on Dec 01, 2020 is 0.000000087210450983 ETH)} \label{tab:gas}
\end{table}


\begin{table}[]
	\begin{tabular}{|l|l|}
		\hline
		code	& Size(bytes)  \\ \hline
		bytecode	& 27178  \\ \hline
		deployed	& 26774  \\ \hline
		initialisation and constructor code 	& 404 \\ \hline
	\end{tabular}
	\caption{Code size}
\end{table}

\begin{table}[]
	\begin{tabular}{|l|l|l|l|l|}
		\hline
		& Alice &  Bob & Trustee1*  & Trustee2  \\ \hline
		Number of Transactions	& 5 &5  &44  &17  \\ \hline
		Total gas cost	& 1246712 &  1246712& 52952603  & 6420996 \\ \hline
		Total cost in USD	&  66.87& 66.87 &2840.41  & 344.43 \\ \hline
	\end{tabular}
	\caption{Cost for each party} 
\end{table}

\begin{table}[]
	\begin{tabular}{|l|l|}
	\hline
	setting	& Total gas  \\ \hline
	1 gate, 2 trustees	& 61867023  \\ \hline
	2 gate, 2 trustees	&  121240622 \\ \hline
	1 gate, 3 trustees	& 68288019 \\ \hline
\end{tabular}
	\caption{Different settings}
\end{table}

\begin{table}[]
	\begin{tabular}{|l|l|l|l|l|l|}
		\hline
		 &  & ~\textbf{Ethereum} & \multicolumn{3}{c|}{~\textbf{Arbitrum}} \\ \hline
		~\textbf{Contract }& ~\textbf{Function}  & ~\textbf{Gas}  &  ~\textbf{L1 Gas}  & ~\textbf{L2 Arbgas} & ~\textbf{Input size (bytes) }\\ \hline 
		\multirow{6}{*}{} 
		&  loadCiphertexts & 304668 & 7869 & 820507& 832\\  \cline{2-6}
		& PET\_subtract & 242131 & 5469& 4789799& 640\\  \cline{2-6}
		&  DHProve\_Trustee1Pf\_Rand& 815340 &  11488& 10972720& 832\\  \cline{2-6}
		PET.sol	&  DHProve\_Trustee2Pf\_Rand& 393561 &11440& 11069485 & 832 \\  \cline{2-6}
		&  DHProve\_Trustee1Pf\_PartialDec& 364298 & 11452& 10692786& 832\\  \cline{2-6}
		& DHProve\_Trustee2Pf\_PartialDec & 363612 & 11512& 10689113& 832\\  \cline{2-6}
		& FullDecryption  &  107086& 6236& 4258675 & 512\\  \hline
	\end{tabular}
	\caption{Comparison with arbitrum} 
\end{table}

\begin{mdframed}
 	\textbf{Mix and match}
 	\newline
 	 \newline
 	\textbf{Parties: } Alice, Bob, Trustee1 (who is also the manager), Trustee2 
 	 \newline
 	\textbf{Input: } Alice's ElGamal ciphertext, Bob's ElGamal ciphertext, encrypted NAND table 
 	\newline
 	\textbf{Output: } Matching value from the table based on Alice and Bob's inputs
 	\newline
 	
	Preparation stage for mix and match are done off-chain:
	\begin{enumerate}
		\item Manager creates the table for NAND gate with two inputs and encrypts each cell using ElGamal. 
		\item Manager shuffles the rows and rerandomizes the cells of the table.
		\item Alice and Bob encrypt their respective inputs $a$ and $b$ as $[a]$ and $[b]$.
	\end{enumerate}

 	The protocol continues as follows:
 	\begin{enumerate}
 		\item Alice and Bob deposit their ciphertexts, $[a]$ and $[b]$, into the smart contract. They also deposit enough funds into the contract. The funds will be used to compensate the work done by the trustees.
 		\item Trustee1 (Manager) subtracts user's (Alice or Bob) input from the cell's value.
 		\item Trustee1 submits blinded value and the values for zero knowledge proof (ZKP).
 		\item Trustee2 submits blinded value and the values for ZKP.
 		\item Trustee1 submits value for partial decryption and the values for ZKP.
 		\item Trustee2 submits value for partial decryption and the values for ZKP.
 		\item Trustee1 determines the matching row, partially decrypts the result and submits ZKP.
 		\item Trustee2  determines the matching result by partially decrypting Trustee1's result and submits ZKP.
 		\item Funds are transferred to the trustees.
 		\item Alice and Bob learn the final result.
 	\end{enumerate}
\end{mdframed}

Steps $2-6$ are repeated for each input cell.

Absentia is a system that performs secure function evaluation on Ethereum. The parties leave their secret, encrypted inputs to the system and the rest is handled in the absence of them (submit-and-go property).

The system is first implemented in Mathematica to have test vectors for the Solidity implementation. For our example, we used AND gate and there are two trustees that perform the computations. One of the trustees is designated as the manager who is responsible for creating and encrypting the table and finding the matching value at the end of the protocol.

The system is composed of two smart contracts and a library to perform elliptic curve operations.

The gas limit in Truffle is set to $0xBD6EE1$.

The elliptic curve library is taken from~\footnote{\url{https://github.com/orbs-network/elliptic-curve-solidity/blob/master/ECops.sol}}. As multiplication is the most expensive function, we do not perform the multiplication itself in the smart contract. Instead, we use the trick explained in~\footnote{\url{https://ethresear.ch/t/you-can-kinda-abuse-ecrecover-to-do-ecmul-in-secp256k1-today/2384}}. Multiplication is preformed off-chain and the function $ecmulVerify$ takes the multiplicand, scalar and the product as input and verifies that the multiplication is done correctly.

Contract size limit is 24KB according to EIP170. As our main contract is over this limit, we ran our test in Truffle by allowing unlimited contract size.

Factory design pattern is used to realize the system. Mix and match contract creates instances of PET contract for each cell in the NAND table , so that Alice and Bob's inputs are compared against related cells.

Depositing values to each PET vs depositing once

How payments are handled? 

Alice and Bob can deposit money any time they want during the protocol. They can also withdraw any excess amount from their balances as long as there is still enough funds left to pay the trustees. They can also withdraw funds either before the protocol starts or after the protocol ends, in the case where they accidentally deposited funds. Otherwise, the funds are kept for a month, until it can be withdrawn again. ~\footnote{\url{https://github.com/OpenZeppelin/openzeppelin-contracts/blob/master/contracts/utils/ReentrancyGuard.sol}} is used to prevent reentrancy attack.

Funds are transferred tot the trustees at the end of the protocol, after thay determined the final output. In another possible design, trustees can be paid at each step they performed towards reaching the final output.

~\subsection{Arbitrum}
~\cite{kalodner2018arbitrum}
rollups on arbitrum

functions are performed off-chain and the result is written back to the chain, which is verifiable

dispute resolution



%\usetikzlibrary{calc}
%
%\resizebox{200pt}{200pt}{%
%
%	\begin{tikzpicture}
%	\coordinate (a) at (0,0);
%	\coordinate (b) at (0,1);
%	\coordinate (c) at (1,0);
%	\coordinate (d) at (1,1);
%	\coordinate (e) at (2,0);
%		\coordinate (f) at (2,1);
%	\draw (a) -- (b)node[pos=1.1,scale=0.25]{Client} 
%	(c) -- (d)node[pos=1.1,scale=0.25]{Server}
%	(e) -- (f)node[pos=1.1,scale=0.25]{Client} ;
%	\draw[-stealth] ($(a)!0.75!(b)$) -- node[above,scale=0.25,midway]{\href{https://www.google.ca/}{Text}}($(c)!0.75!(d)$);
%	\draw[stealth-] ($(a)!0.65!(b)$) -- node[below,scale=0.25,midway]{Hey} ($(c)!0.65!(d)$);
%	\end{tikzpicture}
%	}
%
%\begin{sequencediagram}
%	\newthread{t}{:Thread}
%	\newinst[1]{i}{:Instance}
%	
%	\begin{sdblock}{alt}{[condition]}
%		\begin{call}{t}{if\_true()}{i}{}
%		\end{call}
%		
%
%		
%		\begin{call}{t}{if\_false()}{i}{}
%		\end{call}
%	\end{sdblock}
%\end{sequencediagram}
%
%\begin{sequencediagram}
%\newthread[white]{u}{User}
%\newinst[3]{b}{Browser}
%\newinst[3]{t}{TPM}
%\newinst[3]{p}{TTP}
%
%\begin{call}{u}{Init()}{b}{}
%\end{call}
%
%\begin{call}{u}{\href{https://www.google.ca/}{Text}}{b}{}
%
%\mess{b}{verifyAIKAuthSecret}{t}
%
%\begin{call}{b}{get AIK$_{pub}$}{t}{AIK$_{pub}$}
%\end{call}
%
%\end{call}
%\begin{sdblock}{Loop}{}
%
%\begin{call}{u}{Do Something}{p}{AIK$_{pub}$}
%\end{call}
%\end{sdblock}
%\end{sequencediagram}


\begin{sequencediagram}
	\newthread{t}{:Thread}
	\newinst[1]{i}{:Instance}
	
	\begin{call}{t}{\href{https://kovan.etherscan.io/tx/0x10788165eacbbe25066c163fa0cf7a5af07da32c05af001d0f05ad8946974c6e}{Text}}{i}{return value}
	\end{call}
\end{sequencediagram}