% !TEX root = ../main.tex
\begin{abstract}

This paper describes a blockchain-based approach for secure function evaluation (SFE) in the setting where multiple participants have private inputs (multiparty computation) that no other individual should learn. The emphasis of Absentia is reducing the participants' work to a bare minimum, where they can effectively have the computation performed in their absence and they can trust the result. While we use an SFE protocol (Mix and Match) that can operate perfectly well without a blockchain, the blockchain does add value in at least three important ways: (1) the SFE protocol requires a secure bulletin board and blockchains are the most widely deployed data structure with bulletin board properties (immutability and non-equivocation under reasonable assumptions); (2) blockchains provide a built-in mechanism to financially compensate participants for the work they perform; and (3) a publicly verifiable SFE protocol can be checked by the blockchain network itself, absolving the users of having to verify that the function was executed correctly. We benchmark Absentia on Ethereum. While it is too costly to be practical (a single gate costs thousands of dollars), it sets a research agenda for future improvements. We also alleviate the cost by composing it with Arbitrum, a layer 2 `roll-up' for Ethereum which reduces the  costs by 94\%.

\end{abstract}